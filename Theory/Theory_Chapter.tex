% Activate the following line by filling in the right side. If for example the name of the root file is Main.tex, write
% "...root = Main.tex" if the chapter file is in the same directory, and "...root = ../Main.tex" if the chapter is in a subdirectory.
 
%!TEX root =  ../Thesis.tex

\chapter[Theory]{The Higgs Boson and the Supersymmetry}
\label{chapter:Theory}



\section{Introduction}
The goal of particle physics is to understand the fundamental particles of the universe and their interactions. It's a field that is simultaneously impressively advanced and but with tantalizingly unresolved aspects; in general, progress in the field is a team effort between theorists who propose phenomenologies of new physics and the experimentalists who build and analyze data from their large accelerators and detectors.  

The payoff of decades of work in particle physics is the Standard Model, which describes all the known particles and their interactions.  The Standard Model is one of the most thoroughly tested theories in all of science, and it has yet to give a prediction that is not experimentally borne out--an impressive feat.  At the same time, there are known blind spots in the Standard Model, since it does not include gravity, explain dark matter or dark energy, or account for CP violation.   


\section{The Standard Model}
The Standard Model of particle physics was painstakingly constructed over the 20th century and stands as one of the most thoroughly-verified theories in science.  The Standard Model (SM) is a quantum field theory that incorporates two different types of matter particles, the quarks and the leptons, as well as three fundamental forces and their corresponding particles.  However, as we will see, it has several notable shortcomings that attract considerable attention from both theorists and experimentalists.  

\subsection{Quarks and Leptons}
The quarks and the leptons are perhaps the most familiar subatomic particles, as they are the particles that make up matter.  For example, a hydrogen atom is composed of a proton (three quarks) and an electron (a lepton).  There are six quarks total, three ``up-type'' with an electric charge of +2/3 and three ``down-type'' with charge of -1/3.  There are also three leptons, which are electrically charged and massive (the electron, muon and tau), and three neutrinos, which are electrically neutral and nearly massless (the electron, muon and tau neutrinos).  We can classify the quarks and leptons according to ``generation'', where each generation is composed of one up-type quark, one down-type quark, one lepton, and one neutrino.  The quarks and leptons are summarized in Table \ref{tab:QLTable}.

\begin{table}
	\caption{A summary of the fermions. 	\label{tab:QLTable}}
	\begin{tabular}{| c || c | c | c | p{2cm} |}
%		\multicolumn{3}{c}{Quarks} \\
		\hline
		Generation &  Flavor & Electric Charge & Mass (MeV) & SM Interactions\\
		\hline
		\multirow{4}{*}{1} & up quark \it{(u)} & +2/3 & 2.3 & S, W, EM\\
		    & down quark \it{(d)} & -1/3 & 4.8 & S, W, EM\\
		    & electron \it{(e)}& -1 & 0.511 & W, EM \\
		    & electron neutrino \it{($\nu_{e}$)} & 0 & $<$2.2$\times 10^{-6}$ & W\\
		\hline
		\multirow{2}{*}{2} & charm quark \it{(c)} & +2/3 & 1290 &  S, W, EM \\
		    & strange quark \it{(s)} & -1/3 & 95 & S, W, EM \\
		    & muon \it{($\mu$)} & -1& 105.7 & W, EM \\
		    & muon neutrino \it{($\nu_{\mu}$)} & 0 & $<$0.170 & W \\
		\hline 
		\multirow{2}{*}{3} & top quark \it{(t)} & +2/3 & 173,340 & S, W, EM \\
		    & bottom quark \it{(b)} & -1/3  & 4180 & S, W, EM \\ 
		    & tau \it{($\tau$)} & -1 & 1776 & W, EM\\
		    & tau neutrino \it{($\nu_{\tau}$)} & 0 & $<$15.5 & W\\		    
		\hline
	\end{tabular}
\end{table}


\begin{table}
	\caption{The bosons of the Standard Model: their masses and interactions.  \label{tab:boson_table}}
    \center
	\begin{tabular}{| c || c | c |}
	\hline
	Particle & Associated Force & Mass \\
	\hline
	gluon & strong & massless \\
	photon & electromagnetic & massless \\
	W$^\pm$ & weak & 80.4 GeV \\
	Z & weak & 91.2 GeV \\ 
	\hline
	\end{tabular}
\end{table}


All of the quarks and leptons are fermions, meaning they have half-integer spin.

\subsection{Bosons and Forces}

The forces between fermions are carried by bosons, which are integer spin particles.  There are three forces described in the Standard Model: electromagnetic, weak, and strong.  The electromagnetic force is carried by the photon and describes, for example, electric forces between particles.  Photons are massless and as a result, the electromagnetic field can extend infinitely far.  The weak force is carried by W$^+$, W$^-$ and Z$^0$ bosons.  These particles are massive, which means that they are limited in how far they can travel and thus the weak force is confined to distance scales approximately the size of an atomic nucleus.  The weak force is involved when one type of fermion changes into another type of fermion, for example, when a neutron decays or a nucleus fissions.  The strong force is carried by gluons, which are massless but because of confinement, the strong force is restricted to the nuclear scale.  The strong force is responsible for holding quarks together into protons, neutrons and other hadrons. Last, there is the gravitational force, which we will neglect as it is many orders of magnitude weaker than the other forces under discussion.

The electromagnetic and weak forces, as it turns out, can be unified into a single ``electroweak'' force, as discovered in the middle part of the 20th century \cite{Weinberg}.  The vector bosons acquire mass, which is known as electroweak symmetry breaking, via the Higgs mechanism.  The Higgs mechanism, and the particle which conveys the Higgs field (the Higgs boson), are explained in more detail in further sections.  Further unification of forces, between the electroweak and strong forces, remains an unfinished project in physics but a topic of much research.  

\section{Electroweak Symmetry Breaking and the Higgs Mechanism in the Standard Model}
The Standard Model is defined by its Lagrangian, which is a mathematical formula that encodes all the Standard Model particles and their interactions.  The SM Lagrangian was built piece by piece over many decades, starting with classical field theory and later being generalized to account for relativity, electromagnetism, the strong and weak forces, and the unification of the weak and EM forces (this is not an exhaustive list of the features of the SM Lagrangian, of course).  

Different terms in the SM Lagrangian account for different types of particles.  The fermions, which have spin 1/2, are generally governed by a Lagrangian that we call the Dirac Lagrangian:

\begin{equation}
\mathscr{L}= i\hbar c \bar{\psi} \gamma^\mu \partial_\mu \psi -mc^2 \bar{\psi}\psi
\end{equation}

(Similarly, the Klein-Gordon Lagrangian is used for spin-0 particles, and the Proca Lagrangian for spin-1.)  When the Dirac Lagrangian is plugged into the Euler-Lagrange equations, the equation that results is a quantum field theory equation describing a particle of mass $m$ and spin $\frac{1}{2}$.  

One problem with simply using the Klein-Gordon Lagrangian as-is arises because the Klein-Gordon Lagrangian is not invariant under local phase transformations.  In other words, if the field $\psi$ is multiplied by an exponential term with a space-dependent (``local'') phase, $\psi \rightarrow e^{i\theta(x)} \psi$, then plugging the new $\psi$ into the Euler-Lagrange equations will result in an extra term because of the derivative of $\theta(x)$.  As written, the Dirac Lagrangian is not invariant under local phase transformations.  The fix for this problem is to replace the ordinary derivative with the covariant derivative:

\begin{equation}
\mathscr{D}_\mu \equiv \partial_\mu + i\frac{q}{\hbar c}A_\mu
\end{equation}

The second term in the covariant derivative cancels the extra term from the derivative of $\theta(x)$, and local phase invariance is reinstated.  However, when we added the second term, we added in a new field $A_\mu$ which must also show up in the Lagrangian but be massless ($m_A$=0) in order to preserve the local gauge invariance that we have so carefully constructed.  The trouble here is that, when we think about this formalism being used to describe the weak force, the corresponding $W$ and $Z$ bosons are definitely \textit{not} massless.  What we need is a modification to this procedure that will leave us with at least two massive degrees of freedom, which we can interpret as the $W$ and $Z$, and no massless particle corresponding to $A_\mu$.

Let's now consider what happens when the Lagrangian has a slightly different functional form: instead of a quadratic term $-mc^2\bar{\psi}\psi$, it's perfectly valid to have a quartic term also: $-\mu^2\psi^2+\lambda^2 \psi^4$.  In this case, however, ground state of the field $\psi$ does not occur when $\psi=0$, but rather when $\psi$ takes on a value of $\psi = \pm \mu^2/\lambda$.  There is a symmetry as to whether $\psi$ takes on the positive or negative solution to the equation; the fact that the system must pick one solution or the other means that it undergoes \textit{spontaneous symmetry breaking}.

The Higgs mechanism is the powerful combination of local gauge invariance and spontaneous symmetry breaking, which achieves our goal of getting rid of the massless $A_\mu$ term while allowing $W$ and $Z$ to be massive.  The field $\psi$ is a complex field, with both a real and imaginary part $\psi=\psi_1+i\psi_2$, which we can make locally invariant by the same trick as before (replace the derivative with the covariant derivative, and add in a new field $A_\mu$).  If, however, the Lagrangian has a form that allows for spontaneous symmetry breaking, the field $\psi$ can undergo a change of variables $\eta\equiv \psi_1 - \mu/\lambda$, $\xi\equiv \psi_2$ and the Lagrangian can be rewritten in a way that the massless gauge field $A_\mu$ has now acquired a mass.  The massless particle we first encountered when trying to write a locally-invariant Lagrangian is represented by the $\xi$ term.  A transformation of the $\psi$ field $\psi \rightarrow \psi' = (cos\theta + i sin\theta)(\psi_1+i\psi_2)$, with $\theta=tan^{-1}(\frac{\psi_2}{\psi_1})$ makes $\psi_2'=0$, which gets rid of the massless particle that was troubling us before.  If we generalize this analysis to a  complex scalar doublet, which has four degrees of freedom (this example had 2), we will have 4 massive particles.  Three of these are the gauge bosons, $W^\pm$ and $Z$; the fourth is a new massive scalar particle which we call the Higgs boson.



 
 
 
\section{Supersymmetry}

\begin{centering}
\includegraphics[width=0.6\textwidth]{/Users/caitlinmalone/Documents/Thesis/Theory/FeynmanDiagrams/Higgs_mass_corrections.pdf}\label{fig:higgs_mass_corrections}
\end{centering}

Despite its robustness in the face of experimental scrutiny, the Standard Model has several important shortcomings.  One of the most important is the hierarchy problem, which refers to the quadratic divergence of the Higgs mass via self-coupling.  On the one hand, it is intuitive (and now experimentally verified) that the Higgs mass is of the same order of magnitude as the masses of the electroweak bosons.  On the other hand, the Higgs can couple to itself via fermion-antifermion pairs, which introduces correction terms to the Higgs mass.  For example, the diagram shown in Figure~\ref{fig:higgs_mass_corrections} (a) contributes the following correction term to the mass:

\begin{equation}
	\delta m_H^2 = -\frac{|\lambda_f |^2}{8\pi^2}[\Lambda_{UV}^2+\ldots]
\end{equation}

$\Lambda_{UV}$ is the ultraviolet cutoff scale, the energy at which the Standard Model breaks down and new physics must be enter the picture.  The exact value of $\Lambda_{UV}$ is not known, but a reasonable a priori guess would be the Planck scale, about $10^{19}$ GeV.  However, since the experimentally measured mass of the Higgs is about 125 GeV, there needs to be fine-tuning on the scale of $10^{30}$ GeV for the numbers to come out correctly.

Supersymmetry solves this problem by introducing a mirror set of particles to the Standard Model particles, where each SM boson has a corresponding supersymmetric fermion and each SM fermion has a SUSY boson.  These SUSY particles would also couple to the Higgs, and introduce additional terms to the mass, namely

\begin{equation}
	\delta m_H^2 = \frac{|\lambda_s |^2}{16\pi^2}[\Lambda_{UV}^2-2m_2^2ln(\Lambda_{UV}/m_s)+\ldots]
\end{equation}

The ultraviolet cutoff $\Lambda_{UV}$ enters again here, this time with the opposite sign, so that the two terms end up largely canceling when the Higgs mass is computed.  

Taken at face value, SUSY requires that the supersymmetric partners have the same masses as their SM counterparts.  If this were the case, though, we would have seen some hint of the SUSY particles already either directly or through indirect effects, which leads us to say that supersymmetry is ``broken'' (in the sense that is is not a true symmetry of nature, not in the sense that it doesn't hold theoretically).  When we consider broken supersymmetric theories, the masses of the SUSY particles can become large, and indeed the mass limits for SUSY particles now require that they generally be at least several hundred GeV, up to several TeV.  One problem with this trend is that the supersymmetric solution to the hierarchy problem assumes SUSY partners that are light, less than a TeV or so, so that the SUSY contributions to the Higgs mass are roughly the same size as the SM contributions.  As direct and indirect searches exclude much of the phase space below a few TeV, it becomes harder to find SUSY scenarios that elegantly solve the hierarchy problem and SUSY becomes less appealing in that sense. 

SUSY has additional motivations, however.  Another appealing feature of SUSY is that it adjusts the coupling constants of the three SM forces.  At very high energies the coupling constants come close to the same value, but do not quite match up, a theoretically unsatisfying fact that SUSY addresses.  When SUSY enters the picture, the running values of the coupling constants change such that they unify at a high energy scale.

\begin{figure}
	\includegraphics[width=\textwidth]{/Users/caitlinmalone/Documents/Thesis/Theory/FeynmanDiagrams/3_couplings.pdf}
	\label{fig:couplings}
\end{figure}


A third appealing feature of SUSY is that it provides a natural candidate for dark matter.  From studies of galactic rotational curves, and other astrophysical investigations, it seems clear that there is a significant amount of matter (``dark matter'') floating around the galaxy that does not interact via the electromagnetic or strong nuclear forces, but does interact gravitationally (it is not known whether it interacts via the weak nuclear force; a number of experiments are attempting to detect it via the weak force but so far have not produced clear and unequivocal evidence).  Among the supersymmetric particles would be dark matter candidates--particles such as the (fermionic) neutralinos which can be heavy, and thus provide the gravitational interaction of dark matter.  At the same time, most SUSY scenarios preserve a quantity called R-parity, which effectively means that the lightest supersymmetric particle is stable and thus the dark matter would not decay away to Standard Model particles, nor interact via the strong or weak force.  In short, the neutralinos of supersymmetry are candidates for dark matter, which serves as an attractive feature of the theory. 


One less appealing feature of SUSY is that it's a very unconstrained set of theories--depending on the details of the SUSY version involved, there can be hundreds or thousands (or more) new supersymmetric particles that might enter the pictures.  Similarly, SUSY equations can have many free parameters governing the masses, interactions, etc. of the SUSY sector and it is very difficult, perhaps impossible, to probe all the SUSY phase space.  Phenomenologists address this problem in a number of ways, the most important of which for this thesis is the proposal of the MSSM, or Minimally Supersymmetric Standard Model.  The MSSM makes a number of assumptions about the SUSY parameters and their relationships so as to constrain the number of free parameters to a bare minimum of 19.  

% https://indico.cern.ch/getFile.py/access?contribId=4&resId=1&materialId=slides&confId=282042
% http://inspirehep.net/record/810987?ln=en
\section{Higgs Physics in Supersymmetry}
\label{sec:SUSY_Higgs}
Once the constraints of the MSSM have restricted the SUSY phase space to a more tractable 19 parameters, we can see the impact of SUSY on the Higgs sector.  The MSSM also contains (at least) 5 Higgs bosons on account of the two complex Higgs doublets in the theory (these models are called Two Higgs Doublet Models, or 2HDM).  

%\begin{equation}
%\Phi_1 = \frac{1}{\sqrt{2}}
%\end{equation}


Both Higgs doublets acquire a vacuum expectation value (VEV) with values $v_1$ and $v_2$ respectively.  Unitarity provides an important constraint on the value of $v_1$ and $v_2$, namely that 

\begin{equation}
	v^2 = v_1^2 + v_2^2 = \frac{1}{\sqrt{2}G_F} = 246\ GeV^2
	\label{eq:h_246}
\end{equation}


There is one additional complication to the 2HDM formalism.  In its most general form, the Higgs system has CP-violating couplings and flavor-changing neutral currents (FCNC), the latter of which in particular is tightly constrained by experimental evidence.  The Glashow-Weinberg condition explains that if only one Higgs doublet couples to fermions of a given electric charge, there is no Higgs-induced CP violation or FCNC.  There are four ways that the Glashow-Weinberg condition can be met, and the MSSM is consistent with the so-called ``type II'' Higgs doublet model, where one doublet couples exclusively to up-type quarks and the other couples exclusively to down-type quarks.  Then $v_1$ is the VEV to up-type quarks and and $v_2$ is the VEV to down-type quarks, and while equation~\ref{eq:h_246} constrains their sum, the ratio of the two values is a free parameter of the system and is denoted by $tan(\beta)$:

\begin{equation}
	tan(\beta) = \frac{v_u}{v_d}
\end{equation}




The 2HDM models imply the existence not of one Higgs boson, but of five.  The 5 particles include two CP-even particles, h and H, one CP-odd particle A, and two electrically charged particles $H^\pm$.   This analysis is a search for both the CP-even $H$ and the CP-odd $A$--the CP-even $h$ is assumed to be the Higgs particle found at 126 GeV \cite{PDG-Review}.  The masses of $H/A$ are not known.  While there are at least 19 free parameters in the MSSM, the SUSY Higgs sector is (mostly) governed by only two: $m_A$ and $tan(\beta)$.  As a direct result of this, most interpretations of limits (or signal) are presented in terms of the $m_A$/tan$\beta$ phase space favored or excluded.  


\subsection{H/A Searches in the $bbb$ Final State}
The $H/A$ search being performed in this thesis has two important experimental features:

\begin{itemize}
    \item It is produced in association with one or more $b$-quarks
    \item It decays to a pair of $b$-quarks
\end{itemize}

\begin{figure}[H]
	\caption{Feynman diagrams for leading-order production of $h$ in association with $b$-quarks.  }
	\includegraphics[width=0.55\textwidth]{/Users/caitlinmalone/Documents/Thesis/Theory/FeynmanDiagrams/bbH_FeynmanDiagrams.pdf}		
	\includegraphics[width=0.55\textwidth]{/Users/caitlinmalone/Documents/Thesis/Theory/FeynmanDiagrams/bH_FeynmanDiagrams.pdf}
	\label{fig:fd}
\end{figure}

Technically speaking, there is not a Feynman diagram for $H/A$ production that occurs in association with exactly one $b$-quark; even in the second set of diagrams in Figure~\ref{fig:fd} there is a second $b$-quark that comes from the PDF (parton distribution function) of the second proton.  In practice, the fourth $b$-quark tends to be both soft in \pt and far forward in the detector, making it difficult to see experimentally.  For that reason, we only require the presence of one $b$-quark in addition to the two $b$-quarks coming from the $H/A$ decay, which makes for a $bbb$ final state.



While the behavior of the Higgs system can be complicated to fully map out, there are several general trends that emerge when one examines these parameters:

\begin{itemize}
	\item The cross section for $H/A$ production in association with $b$-quarks increases for higher values of $tan(\beta)$
	\item The branching fraction of $H/A$ to $b$-quarks increases for higher values of $tan(\beta)$
	\item At high $tan(\beta)$, the $H/A\rightarrow b\bar{b}$ branching fraction is nearly constant across a wide range of $m_A$
	\item For a given $tan(\beta)$, the production cross section falls for higher $m_A$
	\item The masses of $H$ and $A$, their kinematics, and the $H/A\rightarrow b\bar{b}$ branching fractions are nearly the same, so that for search purposes, one can treat them as one particle with $\sigma \times BR$ twice that of $H$ or $A$ individually
	\item The inherent width of $H$ and $A$ increase as $m_A$ increases 
\end{itemize}





%-------------------------------------------------
\begin{figure}
	\centering
	\caption{The production cross section of $A$, $H$ and $h$ as a function of $m_A$ for several different values of $tan\beta$. \label{fig:xsec_vs_mass} }
	\includegraphics[width=0.8\textwidth]{/Users/caitlinmalone/Documents/Thesis/Theory/figures/mssm_xsec/AH_xsec_vs_mass.pdf}
\end{figure}




\begin{figure}
	\centering
	\caption{The branching ratio to $b\bar{b}$ of $A$, $H$ and $h$ as a function of $m_A$ for several different values of $tan\beta$. \label{fig:br_vs_mass} }
	\includegraphics[width=0.7\textwidth]{/Users/caitlinmalone/Documents/Thesis/Theory/figures/mssm_xsec/AH_br_vs_mass.pdf}
\end{figure}


\begin{figure}
	\centering
	\caption{The inherent width of $A$, $H$, and $h$ as a function of $m_A$ for tan$\beta$=20. \label{fig:width}}
	\includegraphics[width=0.7\textwidth]{/Users/caitlinmalone/Documents/Thesis/Theory/figures/gamma.pdf}
\end{figure}
%-------------------------------------------------



\begin{figure}
	\centering
	\caption{The favored and excluded region in $m_A/tan\beta$ in the $m_h^{max}$ scenario \cite{Carena-2}. \label{fig:mh_max}}
	\includegraphics[width=0.7\textwidth]{/Users/caitlinmalone/Documents/Thesis/Theory/figures/mh_max.pdf}
\end{figure}


\begin{figure}
	\centering
	\caption{The favored and excluded region in $m_A/tan\beta$ in the $m_h^{mod}$ scenario \cite{Carena-2}. \label{fig:mh_mod}}
	\includegraphics[width=0.7\textwidth]{/Users/caitlinmalone/Documents/Thesis/Theory/figures/mh_mod.pdf}
\end{figure}





An important and subtle point worth highlighting is that, while $A$, $H$, and $H^\pm$ are often called the ``SUSY Higgs bosons'', they are SM particles in the sense that they have R-parity of 1.  Therefore, a search for $A$ and $H$ is different from many other SUSY analyses because all the final state particles are SM particles ($b$ quarks) and there is no missing energy associated with SUSY particles that escape detection.  In fact, the Feynman diagrams for the production of $H$ and $A$ in association with $b$-quarks can also be drawn for $b$-quark associated production of the SM Higgs boson $h$; the most important contribution of SUSY is that, when $tan\beta$ is large, it provides an enhancement factor to the vertices that drives the production cross section up to a magnitude that could be large enough to see in 20 $fb^{-1}$ of proton-proton collisions at $\sqrt{s}$=8 TeV.

However, that does not mean that the SUSY Higgs sector is independent of the details of the supersymmetric parameters.  In particular, it is possible that H/A decay to a pair of SUSY particles, such as charginos or neutralinos.  Depending on the scenario, the branching fraction to SUSY particles could be substantial.  Also important for interpreting any limits is the assumption about the Higgs mixing scenario; the CP-even Higgs eigenstates (H and h) can be expressed in terms of the neutral scalar fields and a mixing angle $\alpha$:

\begin{equation}
    \begin{pmatrix}
         H \\ h  
    \end{pmatrix}
    = 
    \begin{pmatrix}
        cos\alpha & sin\alpha \\
        -sin\alpha & cos\alpha
    \end{pmatrix}
    \begin{pmatrix}
        \phi^0_1 \\ \phi^0_2
    \end{pmatrix}
\end{equation}

There are two common assumptions about the mixing angle: that it is maximal (denoted $m_h^{max}$) or the so-called modified mixing scenario \cite{Carena-2} in which the LHC signal corresponds to the light CP-even Higgs boson h in large parts of the $m_A/tan\beta$ plane.


\subsection{Other Types of 2HDM Scenarios}
As mentioned in Section~\ref{sec:SUSY_Higgs}, the MSSM is a particular type of 2HDM.  The   


\subsection{Constraints on $H/A$}
The discovery in July 2012 of an SM-like Higgs boson provides important constraints of the MSSM Higgs sector, but leaves other aspects of the theory tantalizingly unconstrained.  On the one hand, for many values of $m_h$, measuring the exact value of $m_h$ places a strong constraint on $m_A$--so once $h$ is found, one may have a good idea of what the mass of $A$ is, assuming the latter exists.  An important exception is when $m_A$ is much larger than the mass of the SM Higgs boson $m_h$, which is called the decoupling limit, where the properties of $h$ are unaffected by the existence of $H/A$.  This means that for a light $m_h$ below 140 GeV or so (recall that the mass of the discovered particle is about 126 GeV), it is very difficult to ascertain constraints on $H/A$ by studying the properties of $h$.  


 


\section{From Beautiful Theory to Messy Experiment}

It is the experimentalist's job to understand what all this theoretical physics actually looks like it the detector.  This is a deep topic, and one could go into great detail, but a few key trends are outlined here.

First, as we've implied throughout this section, many of the particles for which a physicist searches cannot be seen directly--they live for a fraction of a second, and then decay into other particles.  Those daughter particles often decay themselves, and so on into a  multi-step decay chain.  Only the particles from the last stage in this chain actually get detected, so part of the physicist's job is to reconstruct the original particle(s) from the daughter particles.  Once the daughter particles are identified, all that is needed to reconstruct the parent particle is the transverse momentum (see below), energy, polar angle, and azimuthal angle.  These components go into a Lorentz 4-vector for each particle, and then the Lorentz-vectors can be added together in a relativistically invariant way to get the mass and flight direction of the parent.

Second, certain types of particles leave distinctive signatures.  One important example is that particles associated with QCD, quarks and gluons, are subject to the laws of QCD confinement.  Once they are produced or excited in the hard scatter, quarks and gluons hadronize and shower.  A spray of particles, largely pions, show up in the detector in the general direction of the original parent quark or gluon--if all the shower particles could be collected, they could be recombined into the parent particle.  These showers of particles are called jets, and are often named after the parent particle, such as $b$-jets, light jets and gluon jets.  The topic of jet reconstruction is detailed further in ~\ref{}.  Photons and electrons shower as well, but they generate much narrower electromagnetic showers that are easy to identify and reconstruct, when compared to most QCD showers.  Muons have the cleanest signature of all; because of their large mass, they don't decay before exiting the detector, so they generally make a single charged track that is quite distinctive in the dedicated muon layers of the detector.

A third important feature of particles as they appear in the detector is their $p_T$, or transverse momentum.  Since the initial state of each collision involves two protons, which are composite particles, it is impossible to know the exact longitudinal momentum of two quarks when they collide.  However, the transverse momentum is zero, so in the final state, the transverse momentum must also sum to zero.  When heavy particles are created and then decay, or have other particles deflecting off of them, the $p_T$ of the resulting particles can be tens to hundreds of GeV, which creates a striking signal in the detector and provides an experiment handle when trying to understand the event.








