\documentclass[11pt]{report}
\usepackage{geometry}              % See geometry.pdf to learn the layout options. There are lots.
\usepackage{outline}  
\geometry{letterpaper}                   % ... or a4paper or a5paper or ... 
%\geometry{landscape}                % Activate for for rotated page geometry
%\usepackage[parfill]{parskip}    % Activate to begin paragraphs with an empty line rather than an indent
\usepackage{graphicx}
\usepackage{amssymb}
\usepackage{epstopdf}
\DeclareGraphicsRule{.tif}{png}{.png}{`convert #1 `dirname #1`/`basename #1 .tif`.png}

\title{Outline of Theory Chapter}

%\date{}                                           % Activate to display a given date or no date
\begin{document}
\begin{outline}

	\item The Standard Model
	\begin{outline}
		\item Quarks and leptons
		\begin{outline}
			\item Familiar particles from everyday life
			\item Three families of quarks and leptons (similarities)
			\item Notable differences between families
		\end{outline}
		\item Bosons and forces
		\begin{outline}
			\item Electromagnetism and the photon
			\item Weak force and vector bosons
			\item Quantum Chromodynamics and gluons
			\item Electroweak unification
			\item Toward a GUT or TOE?
		\end{outline}
	\end{outline}
	
	\item Electroweak Symmetry Breaking and the Higgs Mechanism
	\begin{outline}
		\item Standard model Lagrangian and massless vector bosons
		\begin{outline}
			\item Griffiths equations 10.129 and 10.136, with interpretation as new particle
		\end{outline}
		\item Higgs mechanism breaks symmetry and provides mass to particles
		\begin{outline}
			\item W, Z bosons massive now
			\item Yukawa couplings to other particles 
			\item Qualitative interpretation as particles being "slowed down" as they travel through field
		\end{outline}
	\end{outline}
	
	\item Supersymmetry
	\begin{outline}
		\item Problems with the SM as currently formulated
		\begin{outline}
			\item Hierarchy problem: mass imbalances between particles, runaway Higgs mass from self-coupling
			\item Matter-antimatter asymmetry
			\item Does not explain dark matter
			\item No unification of couplings at high energy
		\end{outline}
		\item Basic structure of SUSY
		\begin{outline}
			\item Superpartners of particles
			\item Boson-fermion symmetry
		\end{outline}
		\item SUSY as a solution to problems mentioned above
		\begin{outline}
			\item Natural dark matter candidate
			\item Can tune couplings to converge at high energy scale
			\item Cancellation of terms solves Higgs self-coupling problem
			\item Unitarity problem (look into this more)
			\item Postpones hierarchy and CP violation problems
		\end{outline}
		\item Simplified SUSY scenarios
		\begin{outline}
			\item Many parameters allowed
			\item Constrain certain relationships to make problem more tractable
			\item Free parameters left: tan$\beta$, m$_0$, etc.
		\end{outline}
	\end{outline}
	
	\item Higgs Physics in Supersymmetry
	\begin{outline}
		\item 5 Higgs bosons in SUSY
		\begin{outline}
			\item 2 Higgs doublets with 8 DOF
			\item 3 DOF already taken by massive vector bosons--5 left
			\item 2 charged scalar, CP-odd pseudoscalar A, CP-even scalars h and H
		\end{outline}
		\item Higgs couplings, SUSY as a broken symmetry
		\begin{outline}
			\item Mass as Higgs coupling $\times$ VEV
			\item No SUSY particles seen, so they must be heavy
		\end{outline}
	\end{outline}
	
	\item Higgs Phenomenology in Supersymmetry
	\begin{outline}
		\item Production cross sections
		\begin{outline}
			\item Plots of cross section as a function of mass and tan$\beta$
			\item Degeneracy of h and H/A at high tan$\beta$
			\item Feynman diagrams and interpretation for bH production
		\end{outline}
		\item Branching fractions and widths
		\begin{outline}
			\item Plots of branching fractions in SUSY
			\item Widths as calculated in FeyHiggs (or similar)
			\item Lifetime
		\end{outline}
		\item Limits from other experiments/measurements
	\end{outline}

\end{outline}  
\end{document}