
\documentclass[11pt]{article}
\usepackage{geometry} % see geometry.pdf on how to lay out the page. There's lots.
\geometry{a4paper} % or letter or a5paper or ... etc
\usepackage{amssymb,amsmath}
\usepackage{multirow}
% \geometry{landscape} % rotated page geometry

% See the ``Article customise'' template for come common customisations

\title{}
\author{}
\date{} % delete this line to display the current date

%%% BEGIN DOCUMENT
\begin{document}

\maketitle
\tableofcontents
\label{sec:atlas}

% http://cds.cern.ch/record/331063/files/ATLAS-TDR-4-Volume-I.pdf?version=1
\section{Inner Detector}

The innermost layer of the ATLAS detector is the tracker, which provides precision measurements of the trajectories of charged particles.  As a charged particle traverses the layers of the tracker, it ionizes the detector material which creates small electrical signals that can be amplified and read out by the system.  These so-called ``hits'' are combined during reconstruction into tracks, which represent the paths of particles like electrons, muons, and charged pions in the detector.

The information from the tracker is used in particular to determine the transverse momentum (p$_T$) of charged particles, and to perform particle identification.  The tracker consists of three primary subsystems: the pixels, semiconductor tracker (SCT), and transition radiation tracker (TRT).  A charged particle created at or near the interaction point would typically travel through all three subsystems, creating some number of hits in each one.  On average, a track has 3 pixel hits, 8 hits in the SCT (4 double-sided hits) and about 34 hits in the TRT, with all the inner detector subsystems enclosed by a 2 tesla solenoidal magnetic field. 

%http://cds.cern.ch/record/1290332/files/VERTEX%202010_015.pdf
\subsection{Pixel System}
The pixel system sits physically closest to the beam line and interaction point.  It is built of silicon pixels that measure 50 x 400 micrometers each, which are organized into sensors.  Each sensor contains 46,080 pixels, and the sensors are arranged onto modules.  Each module contains 16 sensors, and there are 1744 modules total in the pixel system, organized into 3 layers each in the barrel and the two endcaps.  The pixel system in aggregate contains 80 million channels and measures 1.4 meters long by 0.5 meters in diameter.

% http://iopscience.iop.org/1748-0221/3/07/P07007/pdf/1748-0221_3_07_P07007.pdf
The pixel technology is designed to give high-precision measurements of the location and momentum of charged tracks.  A pixel module has two main components, the silicon sensor and the front-end chip, which are bump-bonded together.  When a charged particle traverses the sensor it ionizes silicon atoms, creating so-called electron-hole pairs, and then a bias voltage applied across the sensor causes the electrons and holes to drift to opposite sides of the sensor, where they can be read out by the front-end chips.  

The pixel readout is based on detecting and quantifying this ionization current.  A particle with greater momentum will create more electron-hole pairs and thus a longer readout pulse, which passes through a preamplifier and then a discriminator.  The discriminator is set with a tunable threshold number of electrons, typically a few thousand, and the signal metric is then the length of time for which the pulse was above that value (called ``time over threshold'', or TOT).  Having a threshold in place helps distinguish ionization current from leakage current, which occurs when the silicon does not have robust insulator characteristics and current begins to flow across the sensor even when there is no ionizing particle present.  One important effect of radiation damage to the pixel detector is that it damages the silicon, allowing for increases in leakage current, so over the course of large radiation doses to the detector the thresholds sometimes have to be raised to compensate for the damaged material.  

The pixel system all together provides about 15 $\mu m$ transverse and $<$1 mm longitudinal resolution of the beamspot and surrounding area, which serves several important purposes.  First, since there are typically many hard p-p collisions in each bunch crossing, the pixels system enables the reconstruction of multiple primary vertices that are typically separated by \textbf{some distance}.  This is critical for controlling pileup in the high-luminosity LHC environment.  Second, the transverse resolution of \textbf{some amount} enables precision b-tagging for identification of bottom quarks.  B quarks typically have a lifetime of \textbf{lifetime}, and since they are often created in high-p$_T$ collisions or come from the decay of heavy particles, they can have considerable $p_T$ and travel a few millimeters before decaying.  B-tagging algorithms typically look for evidence of secondary decay vertices that are displaced from the beamspot in the transverse direction.   

\subsection{Semiconductor Tracker (SCT)}
Like the pixels, the silicon microstrip tracker (SCT) is a silicon detector, although the geometry is distinctly different from the pixel geometry.  The SCT consists of 8 layers of microstrips organized into 4 double-sided pairs, where the two members of each pair have an offset angle of 40 mrad.  The geometry of the SCT, with substantially larger detector elements, allows for a lower cost and less material in the detector than if the same coverage were implemented in pixels, while maintaining 16 $\mu$m resolution of tracks in r/$\phi$ and 580 $\mu$m resolution in Z.   


\subsection{Transition Radiation Tracker}



 
\subsection{Inner Detector Performance and Tracking}
Measurements from the pixel, SCT and TRT are combined in the track reconstruction.  There is a broad range of properties that tracks might have, so the inner detector has to be able to measure tracks with p$_T$ ranging from 150 MeV to 30 GeV or more.   

The main track reconstruction algorithm used at ATLAS is a version of the Kalman filter algorithm.  The track reconstruction begins with track seeds, which are collections of a few hits in the pixel subsystem that could plausibly be parts of tracks.  The advantage of starting with seeds, rather than attempting to find an entire track at once, is that seeds allow flexibility at the beginning of the search (when the track trajectory is the least known) while keeping computational costs down.  Once a seed has been identified, the Kalman fitter (as the ATLAS algorithm is called) projects where the next hit would be if there is truly a track present, and then looks for the presence of a hit at the predicted location in the next tracker layer.  If a hit is found there, the predicted trajectory of the particle may be refined and the next hit is predicted and sought, until all the detector layers have been traversed. If there is no hit present where one is predicted, the Kalman fitter can project two layers further, to allow for the possibility of a hole where the particle did not leave a hit for some reason. 

\section{Calorimeters}
The ATLAS detector has two calorimeter systems: the electromagnetic (EM) calorimeter, designed to measure the energy of electrons and photons, and the hadronic calorimeter, for measuring the energy of hadrons, jets, tau leptons and missing transverse energy.  

%http://www-library.desy.de/preparch/desy/proc/proc10-01/meng.pdf
%http://iopscience.iop.org/1742-6596/110/9/092007/pdf/jpconf8_110_092007.pdf
\subsection{Electromagnetic Calorimeter}
The electromagnetic (EM) calorimeter measures electrons and photons after they exit the tracking system.  The calorimeter consists of two major parts, the barrel and the endcaps; the barrel measures particles with $|\eta|<$1.475 and the endcaps measure particles with 1.375$<|\eta|<$3.2.  The calorimeter is a sampling calorimeter, with the passive showering material (lead) interleaved with active energy measurement material (liquid argon, or LAr).  An electron or photon will interact with the lead as it travels through it, creating an electromagnetic shower, which then propagates to the adjacent LAr layer, where it is measured and read out.

A notable feature of the EM calorimeter is the accordion geometry, which has several key features.  First, the geometry enables complete coverage in $\phi$ without azimuthal cracks.  Second, the LAr sampling layer between lead layers is constant throughout the calorimeter barrel.  Third, a particle traveling through the calorimeter will generate approximately the same number of sampling instances (i.e. measurements) regardless of the direction in which it travels.  These pieces add up to a very uniform coverage of electromagnetic calorimetry.

The performance requirement for the resolution of the EM calorimeter is $\sigma_E/E=\frac{10\%}{\sqrt{E}}+$0.7\%, and a crucial part of reaching this resolution is precisely understanding the shape of the readout pulse.  The traversing particle produces an electromagnetic shower where the drift time of the particles in the shower cause a readout pulse that is roughly triangular in shape and typically 400 ns long.  This pulse is shaped by the readout electronics and the signal shape is simulated with Monte Carlo and calibrated using precisely known test pulses deposited into the readout chain.  However, as detailed below, the presence of multiple interactions per bunch crossing, known as pileup, has a significant effect in the calorimeters and is an important task for understanding jet energy.

%http://cds.cern.ch/record/1478440/files/ATL-TILECAL-PROC-2012-008.pdf
%http://arxiv.org/pdf/1305.0550v1.pdf
\subsection{Hadronic Calorimeter} 
The hadronic calorimeter measures the energy from hadrons, jets, taus and allows for a measurement of missing transverse energy.  The hadronic calorimeter, nicknamed the tile calorimeter because of its composition of scintillating tiles interleaved with steel plates.  The calorimeter is partitioned into four major subsections, two barrel sections and two extended barrel sections, allowing for measurements out to $|\eta|<$1.7.  In order to stop all particles from the collision, with the notable exception of muons, the hadronic calorimeter is \textbf{some distance} radiation lengths thick.


%http://arxiv.org/pdf/0802.1189.pdf
\section{Jet Reconstruction}
Jet reconstruction is the process of assembling calorimeter deposits together into a physics object, called a jet, that ideally will do a good job of representing the characteristics ($p_T$, energy, flavor) of the quark or gluon that originated the jet.  There are a number of clustering algorithms for assembling the calorimeter cells, and postprocessing steps for improving the performance of jets in analyses--pileup subtraction, energy calibrations, grooming and trimming, to name a few.  

The default jet clustering algorithm in ATLAS is the anti-$k_t$ algorithm with a distance parameter of 0.4.  Roughly summarized, this algorithm starts with a calorimeter cell that has an energy deposit much higher \textbf{how much higher?} than its neighbor cells, and then the surrounding cells are grouped into the jet in a way that prioritizes high energy over close proximity.  The result is that soft deposits get clustered in with hard deposits, rather than clustering amongst themselves.  The distance parameter of 0.4 is a cutoff as to how far away from the seed to look for additional deposits.  

A critical feature of this algorithm, or any jet algorithm, is that it be infrared and collinear safe.  That means when additional `ghost' particles with infinitesimally small energy or infinitesimally close radius are added to the area in and around the jet, the properties of the resulting jet do not change.  


%http://iopscience.iop.org/1742-6596/331/2/022024/pdf/1742-6596_331_2_022024.pdf

\section{Muon System}
The ATLAS muon system is designed to measure the $p_T$ of muons with $p_T>$3 GeV, with 3\% resolution up to $p_T<$250 GeV and 10\% resolution up to 1TeV.  The system is composed of four different detector systems located within and around an air-core toroid magnet with a field of 1 Tesla.  Precision tracking in the barrel is done by Monitored Drift Tubes (MDT) and in the endcap by Cathode Strip Chambers (CSC).  Quick-readout triggering is done in the barrel by Resistive Plate Chambers (RPC) and in the endcap by Thin Gap Chambers (TGC).  

The muon system is often used in conjunction with the tracking of the inner detector, since a muon would be expected to create a track in both systems.  Also of particular interest for this thesis is when the muon system can be used in conjunction with the hadronic calorimeter, since b jets often decay semileptonically.  In this case, the muon from the b decay is often nearly colinear with the jet, so a muon that is ``matched'' with a deposit in the hadronic calorimeter can be used to identify and trigger on b decays.  



%https://twiki.cern.ch/twiki/bin/view/AtlasPublic/LuminosityPublicResults#Data_Taking_Efficiency_and_Pileu
%http://cds.cern.ch/record/1459529
%https://cds.cern.ch/record/1435196/files/ATLAS-CONF-2012-042.pdf
\section{Pileup}
All the detector subsystems are affected by the presence of pileup, which are collisions other than the hard scatter collision in a given bunch crossing.  As the LHC delivers higher luminosity for a given number of proton bunches, the luminosity increase comes at the price of many interactions per bunch crossing, and these softer interactions create extra activity in the detector that tends to make events noisier and more challenging to reconstruct accurately.  In 2012, the mean number of interactions per crossing ranged from about 10 up to about 40.  

The inner detector and tracking provide an important tool for understanding in-time pileup.  In-time pileup is additional soft interactions in the same bunch crossing as the hard scatter.  The tracking allows primary vertex reconstruction with a resolution fine enough in z$_0$, \textbf{put some number here}, to resolve separate primary vertices from each other.   The calorimeters cannot resolve individual primary vertices with such precision, though, so a constant struggle in ATLAS is to measure the calorimeter deposits that come from pileup interactions, and where possible to apply corrections that subtract away pileup contributions to jets from the hard scatter.  On average, each additional pileup vertex in an event adds 370(850) MeV to the p$_T$ of a jet reconstructed with the Anti-kT algorithm with R=0.4(0.6).

In addition to in-time pileup, the calorimeters are prone to out-of-time pileup where the signal in a given in event can be affected by the energy flow of previous collisions because of the calorimeter readout signal shapes.  Out-of-time pileup has the effect of adding an average of 60(210) MeV to central jets, and decreasing the forward jets by 350(470) MeV.  

Corrections for both in-time and out-of-time pileup are derived using Monte Carlo simulation of pileup events overlaid on MC simulation of hard scatter events.  The corrections are then validated in data, in a number of different ways.  Two prominent methods are validation in prompt photon events, which produce a $\gamma$+jet signature where the photon is not affected by pileup.  Then the ratio of the p$_T$ of the two objects can be compared to the p$_T$ ratio found in MC, $p^{jet}_T/p^{ref}_T = p^{jet}_T/p^{\gamma}_T$.

A series of calibrations and corrections are combined in sequence, as detailed in \textbf{future section}.  After these corrections are applied, the systematic bias remaining in the jet p$_T$ measurement is 3\% for jets with $p_T>$ GeV.

\textbf{we'll be coming back to this in the analysis, so this section may change}


\section{Trigger}


\end{document}











