\documentclass[11pt]{report}
\usepackage{geometry}              % See geometry.pdf to learn the layout options. There are lots.
\usepackage{outline}  
\geometry{letterpaper}                   % ... or a4paper or a5paper or ... 
%\geometry{landscape}                % Activate for for rotated page geometry
%\usepackage[parfill]{parskip}    % Activate to begin paragraphs with an empty line rather than an indent
\usepackage{graphicx}
\usepackage{amssymb}
\usepackage{epstopdf}
\DeclareGraphicsRule{.tif}{png}{.png}{`convert #1 `dirname #1`/`basename #1 .tif`.png}

\title{Outline of ATLAS Chapter}

%\date{}                                           % Activate to display a given date or no date
\begin{document}
\begin{outline}

	\item Tracker
	\begin{outline}
		\item Pixels
		\begin{outline}
			\item Technology
			\begin{outline}
				\item Physical basis of charged particle detection by semiconductor
				\item Silicon sensors with front end chips for readout
				\item Optical readout
			\end{outline}
			\item Physical structure
			\begin{outline}
				\item 46,080 pixels per sensor
				\item Sensors generally 50x400 microns each
				\item 16 sensors per module, 1744 modules total
				\item 80 million channels, 1.4 m long, 0.5 m wide
			\end{outline}
			\item Readout
			\begin{outline}
				\item On-detector lasers provide optical readout link
				\item DAQ crates off-detector receive and interpret signals
				\item Failures of lasers and motivation for alternatives
			\end{outline}
			\item Relevance for b-tagging
		\end{outline}
		\item SCT
		% http://www.atlas.ch/sct.html
		\begin{outline}
			\item Technology and structure
			\begin{outline}
				\item Silicon technology also
				\item 4 double-sided layers, 6.36 x 6.40 cm
			\end{outline}
			\item Role in track reconstruction
		\end{outline}
		\item TRT
		\begin{outline}
			\item Technology, structure, readout
			\begin{outline}
				\item Ionization of gas when traversed by charged particle
				\item 2-tier threshold system distinguishes between tracking hits and transition radiation
			\end{outline}
			\item Particle identification role
		\end{outline}
	\end{outline}
	
	\item Calorimeters
	\begin{outline}
		\item Electromagnetic
		\item Hadronic
	\end{outline}
	
	\item Muon System

	
	\item Trigger and Data Acquisition
	% overview reference: http://www.atlas.ch/trigger.html
	\begin{outline}
		\item Three-Layer Trigger System
		\begin{outline}
			\item L1
			\begin{outline}
				\item Muon Stream
				\item EGamma Stream
				\item JetTauEtMiss Stream
			\end{outline}
			\item L2
			\begin{outline}
				\item Region of Interest readout based on L1 objects
				\item More granular
			\end{outline}
			\item Event Filter
			\begin{outline}
				\item Full event reconstruction
				\item Reconstruction time and rate
			\end{outline}
		\end{outline}
		\item Data Recording
		\item Event Reconstruction
	\end{outline}
	
	
%	\item Measuring the Luminosity
%%	\end{outline}


\end{outline}  
\end{document}
