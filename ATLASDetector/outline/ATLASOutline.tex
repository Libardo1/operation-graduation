\documentclass[11pt]{report}
\usepackage{geometry}              % See geometry.pdf to learn the layout options. There are lots.
\usepackage{outline}  
\geometry{letterpaper}                   % ... or a4paper or a5paper or ... 
%\geometry{landscape}                % Activate for for rotated page geometry
%\usepackage[parfill]{parskip}    % Activate to begin paragraphs with an empty line rather than an indent
\usepackage{graphicx}
\usepackage{amssymb}
\usepackage{epstopdf}
\DeclareGraphicsRule{.tif}{png}{.png}{`convert #1 `dirname #1`/`basename #1 .tif`.png}

\title{Outline of ATLAS Chapter}

%\date{}                                           % Activate to display a given date or no date
\begin{document}
\begin{outline}

	\item Tracker
	\begin{outline}
		\item Pixels
		\begin{outline}
			\item Technology
			\begin{outline}
				\item Physical basis of charged particle detection by semiconductor
				\item Silicon sensors with front end chips for readout
				\item Optical readout
			\end{outline}
			\item Physical structure
			\begin{outline}
				\item 46,080 pixels per sensor
				\item Sensors generally 50x400 microns each
				\item 16 sensors per module, 1744 modules total
				\item 80 million channels, 1.4 m long, 0.5 m wide
			\end{outline}
			\item Readout
			\begin{outline}
				\item On-detector lasers provide optical readout link
				\item DAQ crates off-detector receive and interpret signals
				\item Failures of lasers and motivation for alternatives
			\end{outline}
			\item Relevance for b-tagging
		\end{outline}
		\item SCT
		% http://www.atlas.ch/sct.html
		\begin{outline}
			\item Technology and structure
			\begin{outline}
				\item Silicon technology also
				\item 4 double-sided layers, 6.36 x 6.40 cm
			\end{outline}
			\item Role in track reconstruction
		\end{outline}
		\item TRT
		\begin{outline}
			\item Technology, structure, readout
			\begin{outline}
				\item Ionization of gas when traversed by charged particle
				\item 2-tier threshold system distinguishes between tracking hits and transition radiation
			\end{outline}
			\item Particle identification role
		\end{outline}
		\item Track reconstruction and SV reconstruction
		\begin{outline}
			\item Quality cuts
			\item Track Reconstruction algorithms %http://iopscience.iop.org/1742-6596/119/3/032014/pdf/1742-6596_119_3_032014.pdf (NEWT)
			\begin{outline}
				\item Kalman fitter
				\item Track seed finding: road window around a track seed
				\item Leapfrog to further points along trajectory
				\item Ambiguity solving: generally give hits to tracks with higher chi2
				\item Points where problems can arise
				\begin{outline}
					\item Noise hits
					\item Dropped hits
					\item Scatters off detector material
					\item Other stuff: bremsstrahlung, photon conversions
				\end{outline}
			\end{outline}
			\item Secondary Vertex Reconstruction
			\begin{outline}
				\item Significance: used in b-jet identification
				\item determination of signed impact parameter (is IP in front of or behind jet)
			\end{outline}
		\end{outline}
	\end{outline}
	
	\item Calorimeters  
	% http://atlas.web.cern.ch/Atlas/TDR/caloperf/caloperf.html
	\begin{outline}
		\item Introduction
		\begin{outline}
			\item Jets, MET, particle ID
			\item Large range of energies detected
		\end{outline}
		\item Electromagnetic
		\begin{outline}
			\item Purpose
			\begin{outline}
				\item Higgs and heavy VB measurements
				\item Shield hadronic calorimeter from EM jets
			\end{outline}
			\item Specifications
			\begin{outline}
				\item lead and liquid Argon accordion geometry
				\item fine (order 0.01) granularity over 3 sampling layers
				\item out to $\eta$=3.2
				\item about 25 radiation lengths thick
			\end{outline}
			\item Performance
		\end{outline}
		\item Hadronic
		\begin{outline}
			\item Purpose
				\begin{outline}
					\item Measurement of hadronic jets from quarks and gluons
					\item particular attention to b, tau, hadronic W
				\end{outline}
			\item Specifications
				\begin{outline}
					\item Tile scintillator and iron in central region
					\item Liquid argon in more forward region (radiation hard)
					\item Endcap at high $\eta$
					\item 11 interaction lengths thick at $\eta$=0 to reduce punch-through to muons
				\end{outline}
			\item Performance
			\item B-jet detection
		\end{outline}
	\end{outline}
	
	\item Muon System

	
%	\item Trigger and Data Acquisition
	% overview reference: http://www.atlas.ch/trigger.html
%	\begin{outline}
%		\item Three-Layer Trigger System
%		\begin{outline}
%			\item L1
%			%\begin{outline}
%			%	\item Muon Stream
%			%	\item EGamma Stream
%			%	\item JetTauEtMiss Stream
%			%\end{outline}
%			\item L2
%			\begin{outline}
%				\item Region of Interest readout based on L1 objects
%				\item More granular
%			\end{outline}
%			\item Event Filter
%			\begin{outline}
%				\item Full event reconstruction
%				\item Reconstruction time and rate
%			\end{outline}
%		\end{outline}
%		\item Data Recording
%		\item Event Reconstruction
%	\end{outline}
	
	
%	\item Measuring the Luminosity
%%	\end{outline}


\end{outline}  
\end{document}
