% Activate the following line by filling in the right side. If for example the name of the root file is Main.tex, write
% "...root = Main.tex" if the chapter file is in the same directory, and "...root = ../Main.tex" if the chapter is in a subdirectory.
 
%!TEX root =  

\chapter[Interpretation]{Interpretation of Results}

Since the experimental result is phrased in terms of a production cross section
times a branching ratio, an interpretation step is required to draw physics conclusions
about the non-Standard Model Higgs sector. A number of standard benchmark scenarios
are provided by the LHC Higgs Cross Section Working Group \cite{mssm_xsec_wg} which span 
a range of plausible model structures.  The interpretation under these benchmarks
is explained in Section~\ref{sec:interp_benchmark}, while in Section~\ref{sec:interp_new_benchmark} we examine other scenarios
that fall outside the standard benchmarks.

\section{Standard Benchmark Scenarios}
\label{sec:interp_benchmark}
Since there are so many analyses at the LHC that deal, either directly or indirectly,
with non-SM Higgs bosons, a set of standard benchmark scenarios are used to bring
order to the chaos and allow for potentially easier comparison and combination 
of results.  These benchmark scenarios are a joint effort of ATLAS, CMS, and 
theorists.  The benchmarks generally set the parameters of the model under inspection,
and then calculate the production cross sections and branching fractions for that
model; then each analysis can convert its exclusions in cross section times 
branching fraction to an exclusion in (usually) $m_A$/tan$\beta$.  The production
cross sections are calculated using the Santander matching algorithm \cite{santander} 
and generally follow the scenario prescriptions suggested in \cite{Carena-2}.

\subsection{$m_h^{max}$ Scenario}
The $m_h^{max}$ scenario is one of the older and more ``classic'' scenarios for 
SUSY Higgs searches.  It uses the value of $X_t$ that maximizes
$m_h$ for large $m_A$ and a given tan$\beta$ value.  Once the Higgs boson was discovered
around 126 GeV, this scenario fell somewhat out of favor because it tends to have an $m_h$ that
is too high, around 140 GeV.  In the era of LEP searches, of course, it wasn't known
what the mass of the Higgs was, and that was as good a guess as any other.  In addition,
this scenario gives conservative bounds on $m_A$ and tan$\beta$.

In the post-Higgs discovery world, the $m_h^{max}$ scenario still serves as a useful
benchmark for its conservative limits, and for placing comparisons against legacy analyses
that used this scenario in their interpretations.   




\subsection{$m_h^{mod}$ Scenarios}
There are straightforward modifications of the $m_h^{max}$ scenario that allow for 
a larger amount of phase space that is still consistent with $m_h$=126 GeV; one
can reduce the amount of mixing in the stop sector to bring $m_h$ down to a value
that agrees better with experiment.  In this scenario, the sign of $X_t$ can be either
positive or negative, with small changes in the excluded region depending on 
which one is chosen.

The $m_h^{mod+}$ and $m_h^{mod-}$ scenarios are generated with the following setup:
\begin{itemize}
    \item HIGLU for calculating the total Higgs production cross section due to gluon fusion \cite{higlu}
    \item ggH\@NNLO for next to next to leading order gluon fusion calculations \cite{gghnnlo} 
    \item bbH\@NNLO (5FS) for next to next to leading order $b$-quark associated production cross sections with the five-flavor scheme \cite{bbhnnlo}
    \item bbH (4FS) scans for Higgstrahlung off bottom quarks in the 4-flavor scheme \cite{bbh_4fs_1} \cite{bbh_4fs_2}
\end{itemize}





\section{Non-Standard Scenarios}
\label{sec:interp_new_benchmark} 



