% Activate the following line by filling in the right side. If for example the name of the root file is Main.tex, write
% "...root = Main.tex" if the chapter file is in the same directory, and "...root = ../Main.tex" if the chapter is in a subdirectory.
 
%!TEX root =  

\chapter[Interpretation]{Interpretation of Results}

The last component of this search is asking how the sensitivity in terms of 
(the model-independent) quantity of production cross section times branching ratio
can be interpreted as a sensitivity with respect to the MSSM parameters $\tan\beta$
and $m_A$.  This interpretation introduces model dependence, as any interpretation does,
%Since the experimental result is phrased in terms of a production cross section
%times a branching ratio, an interpretation step is required to draw physics conclusions
%about the non-Standard Model Higgs sector. 
and a number of standard benchmark scenarios
are provided by the LHC Higgs Cross Section Working Group \cite{mssm_xsec_wg} which span 
a range of plausible MSSM scenarios including $m_h^{max}$ and $m_h^{mod,\pm}$. 

\section{Standard Benchmark Scenarios}
\label{sec:interp_benchmark}
Since there are many analyses at the LHC that deal, either directly or indirectly,
with non-SM Higgs bosons, a set of standard benchmark scenarios are used to bring
order to the chaos and allow for potentially easier comparison and combination 
of results.  These benchmark scenarios are a joint effort of ATLAS, CMS, and 
theorists.  The benchmarks generally set the parameters of the model under inspection,
and then calculate the production cross sections and branching fractions for that
model; then each analysis can convert its exclusions in cross section times 
branching fraction to an exclusion in (usually) $m_A$/tan$\beta$.  The production
cross sections are calculated using the Santander matching algorithm \cite{santander} 
and generally follow the scenario prescriptions suggested in \cite{Carena-2}.

\subsection{$m_h^{max}$ Scenario}
The $m_h^{max}$ scenario is one of the older and more ``classic'' scenarios for 
SUSY Higgs searches.  It uses the value of $X_t$ that maximizes
$m_h$ for large $m_A$ and a given tan$\beta$ value.  Once the Higgs boson was discovered
around 126 GeV, this scenario fell somewhat out of favor because it tends to have an $m_h$ that
is too high, around 140 GeV.  In the era of LEP searches, of course, it was not yet known
what the mass of the Higgs was, and that was as good a guess as any other.  In addition,
this scenario gives conservative bounds on $m_A$ and tan$\beta$.

In the post-Higgs discovery world, the $m_h^{max}$ scenario still serves as a useful
benchmark for its conservative limits, and for placing comparisons against legacy analyses
that used this scenario in their interpretations.   



\subsection{$m_h^{mod}$ Scenarios}
There are straightforward modifications of the $m_h^{max}$ scenario that allow for 
a larger amount of phase space that is still consistent with $m_h$=126 GeV; one
can reduce the amount of mixing in the stop sector to bring $m_h$ down to a value
that agrees better with experiment.  In this scenario, the sign of $X_t$ can be either
positive or negative, with small changes in the excluded region depending on 
which one is chosen\footnote{the scenario name encodes the sign of $X_t$; $m_h^{mod,+}$
has a positive $X_t$ while $X_t$ is negative for $m_h^{mod,-}$}.

The $m_h^{mod+}$ and $m_h^{mod-}$ scenarios are generated with the following setup:
\begin{itemize}
    \item HIGLU for calculating the total Higgs production cross section due to gluon fusion \cite{higlu}
    \item ggH\@NNLO for next to next to leading order gluon fusion calculations \cite{gghnnlo} 
    \item bbH\@NNLO (5FS) for next to next to leading order $b$-quark associated production cross sections with the five-flavor scheme \cite{bbhnnlo}
    \item bbH (4FS) scans for Higgstrahlung off bottom quarks in the 4-flavor scheme \cite{bbh_4fs_1} \cite{bbh_4fs_2}
\end{itemize}

\textbf{m$_A$/tan$\beta$ exclusion plot for $m^h_{mod}$--to be added}

In the final calculus, this search does not surpass searches for $bA/H\rightarrow b\tau\tau$
in terms of either sensitivity to production cross section times branching fraction, 
or exclusions in $m_A/\tan\beta$.  This is for several reasons--at low $m_A$, the 
search in the $\tau\tau$ final state is advantaged by lower trigger $p_T$ thresholds
on the $\tau$ leptons than can be afforded for $b$-jets; at high $m_A$, the FSR from
the $b$-quarks smears out the Higgs mass peak more severely than radiative effects 
in the $\tau\tau$ search.  Additionally, this analysis has the constant challenge of
combinatorial mismatches, where the jets can be incorrectly chosen when reconstructing $H/A$.

On the other hand, this search channel remains a very interesting one.  The branching
fraction to $b$-quarks is 5-7 times higher than the branching fraction to $\tau$ leptons,
and with more clever triggering in planned in the LHC Run 2\footnote{specifically, topological triggers that look
for the presence of muons in jets, allowing for a rough form of $b$-tagging at L1 of the trigger}, 
this search channel remains an interesting one to continue developing.  Similarly, 
smart offline analysis advances such as a multivariate analysis or kinematic cuts that are 
more carefully tuned for the $m_A$ values under investigation might yield more sensitive limits.
Nonetheless, the search reported here (which is the first known search in this final state and mass range)
reinforces the conclusions draw by the $b\tau\tau$ analysis with respect to $m_A/\tan\beta$,
covers new ground for a search in the $bb\bar{b}$ final state, and lays an important piece of groundwork
for studies in this channel going forward.


\begin{figure}[hbt]
\center
\includegraphics[width=0.9\linewidth]{Interpretation/exclusions.pdf}
\caption{The expected 95\% excluded $(H+A)$ $\sigma\times$BR as a function of $m_A$, 
for 450$<m_A<$800 GeV.
\label{fig:exclusions}}
\end{figure}







