% Activate the following line by filling in the right side. If for example the name of the root file is Main.tex, write
% "...root = Main.tex" if the chapter file is in the same directory, and "...root = ../Main.tex" if the chapter is in a subdirectory.
 
%!TEX root =  ../Thesis.tex

\chapter[Fit Strategy and Results]{Fit Strategy and Results}

Once the data has been sent through the trigger and cuts, it is still expected
that a large number of QCD events will remain, in which there might be a (probably small)
number of signal events.  The role of the fit is to mathematically describe
the distribution of the $m_{bb}$ of all the remaining events, and to enable the
possible extraction of a signal from the background.  

This general strategy is possible because the signal events are coming from a 
resonance, the $H/A$ particle, while the background consists of a smoothly 
falling spectrum owing to the fact that it comes from various QCD processes.
That means that $H/A$, if they are present, will show up as a bump in the $m_{bb}$
spectrum.  Using the signal MC, we define what we expect for the shape of
the signal resonance (the normalization is given to us by nature, and 
is a free parameter that must be extracted if the presence of signal is identified),
while the background fit is completely data-driven.  The $m_{bb}$ ranges above
and below the signal resonance, the mass sidebands, are used to 
fit the background distribution to a polynomial, which can then be interpolated
into the signal region.  That allows us to test the goodness-of-fit for a 
background-only hypothesis against the fit quality for a signal+background
hypothesis, for various signal normalizations.  These fit results are then
the ingredients for the limit-setting procedure, which is detailed in the next section.

This chapter details the various categories that are used in the fit, and
the parametric forms that are used to fit them.   

\subsection{Fit Model and Categories}
\label{subsec:fitmodel}
The fit is performed in several different categories, which vary in the 
signal/background ratio, the signal and background shapes, and the
absolute normalizations of the signal and background.  In a situation like
this, where there are several categories that can be defined and the
search sensitivity varies depending on which category is being examined,
it can benefit the overall search sensitivity to fit each category 
separately and then combine them at the end.  While the parameteric form
of the fit is the same in each category (background fit with a polynomial, 
signal fit with a type of bifurcated Gaussian), the categories are fit
separately and so the exact values of the parameters found by the fit
will be slightly different.  


The statistical analysis of the data employs an unbinned likelihood
function, defined as:
\begin{equation}
\text{Pois}(N|\mu S+B) \prod_{k=1}^{njet\ cat} \prod_{l=1}^{N_{tag\ cat}} \prod_{i=1}^{N_{l}} \left[ \mu N_{S,k,l} PDF_{sig,k,l}(m_{bb,i}) + N_{B,k,l} PDF_{bkg}(m_{bb,i}) \right]
\end{equation}
where:
\begin{itemize}
\item the product is over the $b$-tag categories $l$, the n$_{jets}$ categores $k$, and over the events in each category $i$;
\item $N_{S,l}$ and $N_{B,l}$ are the expected signal and background yield in each category;
\item $PDF_{sig,l}(m_{bb})$ and $PDF_{bkg}(m_{bb})$ are the signal and background probability density functions for  the different categories;
\item $\mu$ is a signal strength which multiplies the overall signal prediction.
\end{itemize}


The $b$-tag fit categories $l$ are three exclusive categories to which events are assigned
based on the $b$-tag value of the third-most $b$-jet-like jet in the event (the two most
$b$-like jets have already been $b$-tagged in the trigger, and are assumed to be true
$b$-jets).  Then the categorization in $b$-tag fit categories splits the sample into three
regions with different signal and background enrichments:


\begin{itemize}
    \item $bbb$: one or more jets (in addition to the two triple-tagged jets) passing a tight (60\% working point)
 MV1 cut–-in effect, the full signal selection criteria
    \item $bbloose$: events failing the bbb classification but which have one or more jets passing a loose (80\%
working point) MV1 cut
    \item $bbanti$: events that have no jets passing an 80\% MV1 cut-–effectively a veto on the presence of any
b-tagged jets other than those firing the trigger
\end{itemize}

When assigning events to one of these categories, we only allow $b$-tags on the leading
five jets to count toward the $bbb$ or $bbloose$ categories.  In other words, if the third
$b$-tagged jet in an event is the 6th jet (in $p_T$ ordering) overall, the event 
will be classified as $bbanti$.  This requirement is motivated by our physics awareness that
the $b$-jets coming from signal events should be fairly high $p_T$, so this should have
a minimal effect of rejecting signal that would otherwise be accepted.  On the other
hand, with high-jet-multiplicity QCD events, there is the combinatorial effect of looking across
more jets for $b$-tags that increases the likelihood of an event being classified as signal-like
(bbb or bbloose) as there are more jets in the event.  Only looking at the leading 5 jets
for $b$-tags keeps this effect under control. 


The second type of categorization is based on the number of 
jets in the event: 3, 4, or 5 or more jets.  We find that the signal
shape can change based on the number of events, as well as the overall
signal and background normalizations.  For more details on the effect of 
the number of jets on the signal distributions, see Section~\ref{sec:n_jets_sig}. 




\subsection{PDF Shapes and Fit Constraints}
\label{sec:pdfs}

The background is fit with Bernstein polynomial.  A Bernstein polynomial of degree $n$ is
defined by

\begin{equation}
B_{i,n}(t) = n\choose{i} t^i (1-t)^{n-i}
\end{equation}

for $i=0,1,...,n$, where

\begin{equation}
n\choose{i} = \frac{n!}{i!(n-i)!}
\end{equation}

and there are $n+1$ polynomials of degree $n$.  Bernstein polynomials have the nice property
that they are guaranteed to always have positive values, as well as a reputation for
converging more easily than some other families of polynomials.

The exact degree of the polynomial is a free parameter to be chosen.  There is a tradeoff
between a degree that is too low to capture the full shape of the background, or for which
convergence is a struggle, versus a polynomial with a degree high enough that it can 
``fit anything'' as part of the background, including a possible signal.  For either 
extreme, one also needs to guard carefully against the possibility of the background fit
introducing spurious signal where there is, in fact, no signal to be found.   


%http://www.slac.stanford.edu/econf/C0303241/proc/pres/502.PDF

\begin{figure}[phtb!]
  \begin{center}
  \begin{subfigure}[$m_{A}=400$ GeV]{0.4\textwidth}\includegraphics[width=\textwidth]{FitResults/images/fitMC_bAbb400_1.png}\end{subfigure}
  \begin{subfigure}[$m_{A}=450$ GeV]{0.4\textwidth}\includegraphics[width=\textwidth]{FitResults/images/fitMC_bAbb450_1.png}\end{subfigure}
  \begin{subfigure}[$m_{A}=500$ GeV]{0.4\textwidth}\includegraphics[width=\textwidth]{FitResults/images/fitMC_bAbb500_1.png}\end{subfigure}
  \begin{subfigure}[$m_{A}=550$ GeV]{0.4\textwidth}\includegraphics[width=\textwidth]{FitResults/images/fitMC_bAbb550_1.png}\end{subfigure}
  \begin{subfigure}[$m_{A}=600$ GeV]{0.4\textwidth}\includegraphics[width=\textwidth]{FitResults/images/fitMC_bAbb600_1.png}\end{subfigure}
  \begin{subfigure}[$m_{A}=650$ GeV]{0.4\textwidth}\includegraphics[width=\textwidth]{FitResults/images/fitMC_bAbb650_1.png}\end{subfigure}
  \begin{subfigure}[$m_{A}=700$ GeV]{0.4\textwidth}\includegraphics[width=\textwidth]{FitResults/images/fitMC_bAbb700_1.png}\end{subfigure}
  \begin{subfigure}[$m_{A}=800$ GeV]{0.4\textwidth}\includegraphics[width=\textwidth]{FitResults/images/fitMC_bAbb800_1.png}\end{subfigure}
  \caption{Signal PDFs for $m_{bb}$ in the {\it bbb} category, for events with 3 jets, for different $H/A$ masses. \label{fig:signalPDFs_3j}}
    \end{center}
\end{figure}


\begin{figure}[phtb!]
  \begin{center}
  \begin{subfigure}[$m_{A}=400$ GeV]{0.4\textwidth}\includegraphics[width=\textwidth]{FitResults/images/fitMC_bAbb400_2.png}\end{subfigure}
  \begin{subfigure}[$m_{A}=450$ GeV]{0.4\textwidth}\includegraphics[width=\textwidth]{FitResults/images/fitMC_bAbb450_2.png}\end{subfigure}
  \begin{subfigure}[$m_{A}=500$ GeV]{0.4\textwidth}\includegraphics[width=\textwidth]{FitResults/images/fitMC_bAbb500_2.png}\end{subfigure}
  \begin{subfigure}[$m_{A}=550$ GeV]{0.4\textwidth}\includegraphics[width=\textwidth]{FitResults/images/fitMC_bAbb550_2.png}\end{subfigure}
  \begin{subfigure}[$m_{A}=600$ GeV]{0.4\textwidth}\includegraphics[width=\textwidth]{FitResults/images/fitMC_bAbb600_2.png}\end{subfigure}
  \begin{subfigure}[$m_{A}=650$ GeV]{0.4\textwidth}\includegraphics[width=\textwidth]{FitResults/images/fitMC_bAbb650_2.png}\end{subfigure}
  \begin{subfigure}[$m_{A}=700$ GeV]{0.4\textwidth}\includegraphics[width=\textwidth]{FitResults/images/fitMC_bAbb700_2.png}\end{subfigure}
  \begin{subfigure}[$m_{A}=800$ GeV]{0.4\textwidth}\includegraphics[width=\textwidth]{FitResults/images/fitMC_bAbb800_2.png}\end{subfigure}
  \caption{Signal PDFs for $m_{bb}$ in the {\it bbb} category, for events with 4 jets, for different $H/A$ masses. \label{fig:signalPDFs_4j}}
    \end{center}
\end{figure}


\begin{figure}[phtb!]
  \begin{center}
  \begin{subfigure}[$m_{A}=400$ GeV]{0.4\textwidth}\includegraphics[width=\textwidth]{FitResults/images/fitMC_bAbb400_3.png}\end{subfigure}
  \begin{subfigure}[$m_{A}=450$ GeV]{0.4\textwidth}\includegraphics[width=\textwidth]{FitResults/images/fitMC_bAbb450_3.png}\end{subfigure}
  \begin{subfigure}[$m_{A}=500$ GeV]{0.4\textwidth}\includegraphics[width=\textwidth]{FitResults/images/fitMC_bAbb500_3.png}\end{subfigure}
  \begin{subfigure}[$m_{A}=550$ GeV]{0.4\textwidth}\includegraphics[width=\textwidth]{FitResults/images/fitMC_bAbb550_3.png}\end{subfigure}
  \begin{subfigure}[$m_{A}=600$ GeV]{0.4\textwidth}\includegraphics[width=\textwidth]{FitResults/images/fitMC_bAbb600_3.png}\end{subfigure}
  \begin{subfigure}[$m_{A}=650$ GeV]{0.4\textwidth}\includegraphics[width=\textwidth]{FitResults/images/fitMC_bAbb650_3.png}\end{subfigure}
  \begin{subfigure}[$m_{A}=700$ GeV]{0.4\textwidth}\includegraphics[width=\textwidth]{FitResults/images/fitMC_bAbb700_3.png}\end{subfigure}
  \begin{subfigure}[$m_{A}=800$ GeV]{0.4\textwidth}\includegraphics[width=\textwidth]{FitResults/images/fitMC_bAbb800_3.png}\end{subfigure}
  \caption{Signal PDFs for $m_{bb}$ in the {\it bbb} category, for events with 5 or more jets, for different $H/A$ masses. \label{fig:signalPDFs_5j}}
    \end{center}
\end{figure}



\section{Fit Yields}
\begin{table}
\caption{The $m_{bb}$ windows, predicted yields, and integrals for each of the
    mass points quantified with the fit. \label{tab:bkg_fit_yields}}
    \begin{tabular}{ c c c c c }
        $m_A$ & window low edge & window high edge & prediction & integral \\
    \end{tabular}
\end{table}




